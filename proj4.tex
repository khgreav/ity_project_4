\documentclass[11pt,a4paper]{article}
\usepackage[utf8]{inputenc}
\usepackage[text={170mm,240mm},left=20mm,top=30mm]{geometry}
\usepackage[czech]{babel}
\usepackage[unicode]{hyperref}
\usepackage{times}
\usepackage{url}
\def\UrlBreaks{\do\/\do-}
\begin{document}

\begin{titlepage}
\begin{center}
{\Huge \textsc{Vysoké učení technické v Brně\\[0,3em]{\huge Fakulta informačních technologií}}}\\
\vspace{\stretch{0.382}}
{{\LARGE Typografie a publikování - 4. projekt}\\[0,4em]{\Huge Bibliografické citace}}\\
\vspace{\stretch{0.618}}
\end{center}
{\Large \today \hfill Karel Hanák}
\end{titlepage}

\section{Typografie}
\subsection{Úvod}
Johannes Gutenberg je považován za otce typografie od roku 1440 kdy vynalezl knihtisk. \cite{haley} Typografie se stala neodmyslitelnou součástí tvorby a úpravy textu. Základem všeho je však písmo. 

\subsection{Písmo}
První písmo se datuje k 4. tisíciletí př.n.l kdy se poprvé vyskytlo písmo klínové. Mezi další historicky významná písma patří také egyptské hieroglifické, hieratické a démotické písmo. \cite{citarna} Písmo má mnoho důležitých vlastností. Patří mezi ně například anatomie, rodina, řez, druh a stupeň písma. \cite{scribus}

\subsection{Principy dobrého dokumentu}
Cílem každého dokumentu by mělo být předat informace. Dobře vysázený dokument se lépe čte a informace v něm se tak spíše dostanou ke čtenáři. Dobrý dokument je charakterizován informacemi, strukturou, jednotností a estetikou. \cite{cstug} V jedné typografické studii se poslaly životopisy s různými styly vysázení k ohodnocení a bylo zjištěno, že styly které se zdály být vhodné pro dané odvětví působily lépe. \cite{pov} Každý dokument by měl mít svůj osobitý styl, výběr správného fontu doprovází i vhodně zvolené rozvržení stránky. \cite{rmets}

\subsection{\TeX}
Autorem programu \TeX je Donald Ervin Knuth, profesor na Standfordské univerzitě.
\TeX byl navržen tak aby fungoval nezávisle na operačním systému, je to tedy vhodný nástroj pro sazbu dokumentů díky multiplatformnosti a preciznosti. \cite{beranova}

\subsection{Nádstavba \LaTeX}
\LaTeX je software pro sázení dokumentů, nebo také značkovací jazyk pro dokumenty. Je to nádstavba sázecího programu \TeX. \LaTeX se používá převážně k sázení vědeckých, technických a odborných dokumentů. \cite{kottwitz}

\subsection{Jak to funguje}
Nejprve je třeba vytvořit zdrojový soubor *.tex, pokud v něm není chyba tak se pomocí překladače tex či latex přeloží do souboru *.dvi, tento soubor je dále potřeba převést do formátu *.pdf. Alternativně je možno použít překladače pdftex či pdflatex, v mnoha věcech jsou výhodnější, ale výsledné soubory jsou větší. \cite{bojko}

\subsection{Doporučené editory}
Mezi nejpopulárnější offline editory patří TeXMaker, TeXstudio a TeXworks. Dále existuje velmi populární online editor ShareLaTeX a pro milovníky Vim-u se najde i LaTeX-suite balíček pro Vim. Všechny zmíněné offline editory jsou multiplatformní. \cite{beebom}

\newpage
\renewcommand{\refname}{Použitá literatura:}
\bibliographystyle{czechiso}
\bibliography{sources}
\end{document}